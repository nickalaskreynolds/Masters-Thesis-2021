
%#%#%#%#%#%#%#%#%#%#%#%#%#%#%#%#%#%#%#%#%#%#%#%#%#%#%#
\section{Summary}\label{sec:summary}
We present the highest \gls{sensitivity} and \gls{resolution} observations tracing the disk \glspl{kinematic} toward L1448 IRS3B and IRS3A to date, \cso/\ceo\space comparison: \ab5$\times$\space higher \acrshort{snr} at 4.0~\kms, \ab3$\times$\space higher resolution, and \ab2$\times$\space better velocity resolution as compared to \citet{2016Natur.538..483T}. Our observations resolve three dust \gls{continuum} sources within the circum-multiple disk with spiral structure and trace the kinematic structures using \cso, \htcn/\sot, and \htcop\space surrounding the proto-multiple sources. The central gravitating mass in IRS3B, near -a and -b, dominates the potential as shown by the organized rotation in \cso\space emission. We compare the high fidelity observations with radiative transfer models of the line emission components of the disk. The presence of the tertiary source within the circum-multiple disk, detection of dust continuum spiral arms, and the Toomre~Q analysis are indicative of the disk around IRS3B being gravitationally unstable.

% SN calculation: RMS tobin 1.28E-2, peak 5.8E-2, Nick rms 2.21e-3, peak 4.61e-2, 20 vs 4.5

We summarize our empirical and modeled results:

\begin{enumerate}
    \item We resolve the spiral arm structure of IRS3B with high fidelity and observed IRS3B-c, the tertiary, to be embedded within one of the spiral arms. Furthermore, a possible symmetric inner disk and inner depression is marginally resolved near IRS3B-ab. IRS3B-b may be a high density clump just outside of the inner disk. We also marginally resolve possible spiral substructure in the disk of IRS3A. We calculate the mass of the disk surrounding IRS3B to be \ab0.29~\solm\space with \ab0.07~\solm\space surrounding the tertiary companion, IRS3B-c. IRS3A has a disk mass of \ab0.04~\solm.
    \item We found that the \cso\space emission is indicative of Keplerian rotation at the scale of the continuum disk, and fit a central mass of 1.15$^{+0.09}_{-0.09}$~\solm\space for IRS3B using a fit to the PV diagram. \htcop\space traces the larger structure, corresponding to the outer disk and inner envelope for IRS3B. Meanwhile, the \htcn/\sot\space blended line most likely reflects \sot\space emission, tracing outflow launch locations near IRS3B-c. The \pdspy\space modeling of IRS3B finds a mass of $1.19^{+0.13}_{-0.07}$~\solm, comparable to the PV diagram fit of 1.15$^{+0.09}_{-0.09}$~\solm.
    \item We find that the tertiary companion is forming a central protostar that is less than 0.2~\msun. This upper limit is based on its lack of significant disturbance of the disk kinematics. Moreover, we find that there is a jet originating from the clump, confirming that a protostar is present.
    %\item For IRS3A, the \htcn/\sot\space emission likely reflects \htcn\space emission due to a consistent velocity with \cso. \htcn\space emission indicates Keplerian rotation at the scale of the continuum disk corresponding to a central mass of 1.4~\solm. The molecular line, \cso, is also detected but is much fainter in the source but consistent with a central mass results of 1.4~\solm. The \pdspy\space modeling fit for IRS3A yields mass $1.51^{+0.06}_{-0.07}$~\solm\space which is also comparable to the PV diagram estimate of 1.4~\solm.
    \item The azimuthally averaged radial surface density profiles enable us to analyze the gravitational stability as a function of radius for the disks of IRS3B and IRS3A. We find the circum-multiple disk of IRS3B is gravitationally unstable (Q $<$ 1) for radii $>$ 120~au. We find the protostellar disk of IRS3A is gravitationally stable (Q $>$5) for the entire disk. We marginally detect substructure in IRS3A, but at our resolution, we cannot definitely differentiate between spiral structure and a gap in the disk. If the substructure is spiral arms due to gravitational instabilities, then the disk mass must be underestimated by a factor of 2-4 from our Toomre~Q analysis.
\end{enumerate}

Through the presented analysis, we determine the most probable formation pathway for the IRS3B and its spiral structure, is through the self-gravity and fragmentation of its massive disk. The larger IRS3A/B system (including the even wider companion L1448 NW) likely formed via turbulent fragmentation of the core during the early core collapse, as evidenced by the nearly orthogonal disk orientation and different system velocity for IRS3A and IRS3B.

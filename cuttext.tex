
%#%#%#%#%#%#%#%#%#%#%#%#%#%#%#%#%#%#%#%#%#%#%#%#%#%#%#
\subsection{Protostar Masses}
Comparing the masses of IRS3A (1.51~\solm) and IRS3B (1.15~\solm) to the initial mass function (IMF) \citep[young cluster IMF towards binaries;][]{2005ASSL..327...41C} shows these protostars will probably enter the main sequence as typical, stars once mass accretion from the infalling envelope and massive disks completes. IRS3B-a and -b are likely to continue accreting matter from the disk and envelope and grow substantially in size. 

In addition to the symmetry in the inner clumps, further analysis towards IRS3B-ab of the spatial location of the kinematic centers indicate that the kinematic center is consistent with being centered on the deficit (``deficit''; Figure~\ref{fig:zoomincont}) with a surrounding inner disk, where IRS3B-a is a bright clump moving into the inner disk. The continuum source IRS3B-b would be just outside the possible inner disk radius. These various kinematic centers are within one resolving element of the \cso\space beam, and thus we are unable to break the degeneracy of the results from these observations alone and the central protostar is not apparent from dust emission in our observations.

If we assume the IRS3B-ab clumps surround a single central source, this source would most likely form an A-type (M$_{*}\approx1.6-2.4~$~\solm) star, depending on the efficiency of accretion \citep[10-15\%;][]{2007ApJ...656..293J}. Similarly, IRS3A is likely to form an A-type star. If the IRS3B-ab clumps each represent a forming protostar, then each source would most likely form a F or G-type (M$_{*}\approx0.8-1.4~$~\solm) star depending on the ratio of the masses between the IRS3B-a and IRS3B-b components. IRS3B-c, while currently estimated to have a mass $<0.2$~\solm, it could still accrete a substantial amount mass of the disk and limit the accretion onto the central IRS3B-ab sources. This mechanism can operate without the need to open a gap \citep[][]{1996LNP...465..115A}, which remains unobserved in these systems.

More recently, \citet{2020AA...635A..15M}\space targeted several (7) Class 0 protostars in Perseus with marginal resolution and sensitivities, to fit the molecular lines emission against Keplerian curves to derive protostellar masses. Their fitting method is similar to our own PV diagram fitting and has an average protostellar mass of \ab0.5~\msun. If IRS3B-ab is a single source protostar, then this source would be significantly higher mass (\mstar\ab1.2~\msun) than the average mass of the sample, similar for IRS3A (\mstar\ab1.4~\msun). However, if IRS3B-ab is a multiple protostellar source of two equally mass protostars (\mstar\ab0.56~\msun), then these sources would be consistent with the survey's average protostellar mass.

\citet{2020AA...635A..15M}\space included IRS3B (labeled L1448-NB), using the molecules \tco, \ceo, and SO, and the protostellar parameters are consistent with the results we derived here (\mstar\ab1.4~\msun, \textit{PA}\ab29.5\deg, and \textit{i}\ab45\deg), despite lower sensitivities and resolutions compared to our observations.

\citet[][]{2017ApJ...834..178Y} targeted several well known Class 0 protostar and compared their stellar properties against other well known sources (see reference Table 5 and Figure 10), to determine the star/disk evolution. They derived an empirical power-law relation for Class 0 towards their observations R$_{d}=(44\pm8)\times\big(\frac{M_{*}}{0.1~M_{\odot}}\big)^{0.8\pm0.14}$~au and a Class 0$+$I relation of R$_{d}=(161\pm16)\times\big(\frac{M_{*}}{1.0~M_{\odot}}\big)^{0.24\pm0.12}$~au. The L1448 IRS3B system, with a combined mass \ab1.15~\solm, disk mass of \ab0.29~\solm, and a FWHM Keplerian gaseous disk radius of \ab300~au, positions the target well into the Class 0 stage (\ab245-500~au for the \citet[][]{2017ApJ...834..178Y} relation) and \ab2-3$\times$\space the average stellar mass and radius of these other well known targets. The protostellar mass of IRS3B is larger relative to the sample of protostars observed in \citet[][]{2017ApJ...834..178Y}, which had typical central masses 0.2 to 0.5~\solm. However, this is the combined mass of the inner binary and each component could have a lower mass. L1448 IRS3A, which has a much more compact disk (FWHM Keplerian disk radius of \ab158~au) and a higher central mass than IRS3B (\ab1.4~\solm), is more indicative of a Class I source using these diagnostics. We note there is substantial scatter in the empirically derived relations \citep{2020ApJ...890..130T}, thus the true correspondence of disk radii to an evolutionary state of the YSOs is highly uncertain and we observe no evidence for an evolutionary trend with disk radii.


% ranges from 43\% \citep{2019AA...621A..76M}\space to

%#%#%#%#%#%#%#%#%#%#%#%#%#%#%#%#%#%#%#%#%#%#%#%#%#%#%#
\subsection{Gravitational Potential Energy of IRS3B-c}\label{sec:gpe}
In analyzing the gravitational stability of the IRS3B circumstellar disk, we can also analyze the stability of the clump surrounding IRS3B-c. If the clump around IRS3B-c is sub-virial (i.e., not supported by thermal gas pressure) it would be likely unstable to gravitational collapse, undergoing rapid (dynamical timescale, $\tau_{dyn}$) collapse resulting in elevated accretion rates compared to the collapse of virialized clumps Additionally, it would be unlikely to observe this short-lived state during the first orbit of the clump. Dust clumps embedded within protostellar disks are expected to quickly (t$<10^{5}-yr$) migrate from their initial position to a quasi-stable orbit much closer to the parent star \citep{2019AA...631A...1V}. Thus observing the IRS3B-c clump at the wide separation within the disk is likely due to it recently forming in-situ. The virial theorem states $2E_{kin} + E_{pot}=0$, or in other words we can define an $\mathcal{R}$\space such that $\mathcal{R} := \frac{2E_{kin}}{|E_{pot}|}$\space will be $<1$\space for a gravitationally collapsing clump and $>1$\space for a clump to undergo expansion. Assuming the ideal gas scenario of N particles, we arrive at $E_{kin} = 1.5Nk_{b}T_{clump}$\space where k is the Planck constant and T$_{clump}$ is the average temperature of the particles. The potential energy takes the classic form $E_{pot} = \frac{-3}{5}\frac{GM^{2}_{clump}}{R_{clump}}$. We can define $N=\frac{M_{clump}}{\mu m_{H}}$\space where $\mu$\space is the mean molecular weight (2.37) and m$_{H}$\space is the mass of hydrogen. Assuming the clump is thermalized to the T$_{peak}=54.6$~K, the mass of the clump is $0.07$~\solm, the upper bound for the IRS3B-c protostar is $0.2$~\solm, and the diameter is 78.5~au (Table~\ref{table:obssummary3}), we calculate $\mathcal{R} \approx1.4$\space for the dust clump alone (this $\mathcal{R}$\space is likely an upper bound since our mass estimate for the dust is likely a lower limit due to the high optical depths) and $\approx0.3$\space for the combined dust clump and protostar (This $\mathcal{R}$\space is likely an lower bound since our mass estimate for the protostar an upper limit due to be consistent with the kinematic observations.). This is indicative that the core could be virialized but could also reflect a circumstellar accretion disk around IRS3B-c, or in the upper limit of the protostellar mass, could undergo contraction.
% from nkrpy import constants as c
% (3. / 2. * ((7.33216033e-02 + 0.2) * c.msun) / (2.37 * c.mh) * c.kb * 54.6)         # = 1.5498615001028157e+42 ergs
% (3. / 5. * c.g * ((7.33216033e-02 + 0.2) * c.msun) ** 2 / (78.5 * c.au))  # = 1.0075534802341464e+43 ergs


%#%#%#%#%#%#%#%#%#%#%#%#%#%#%#%#%#%#%#%#%#%#%#%#%#%#%#
\subsection{Mass Accretion}\label{sec:massacc}
The mass in the circumstellar disks and envelopes provide a reservoir for additional mass transfer onto the protostars. However, this mass accretion can be reduced by mass outflow due to protostellar winds, thus we need to determine the maximal mass transport rate of the system to determine if winds are needed to carry away momentum \citep{1998ApJ...502..661W}. While these observations do not place a direct constraint on \mdot, from our constraints on M$_{*}$\space and the observed total luminosity we can estimate the mass accretion rate. In a viscous, accreting disk, the total luminosity is the sum of the stellar and accretion luminosity:
\begin{equation}
L_{bol}\sim L_{*}+L_{acc}
\end{equation}
and the L$_{acc}$\space is:
\begin{equation}
L_{acc}=\frac{GM_{*}\dot{M}}{R_{*}}
\end{equation}
half of which is liberated through the accretion disk and half emitted from the stellar surface. From our observations, we can directly constrain the stellar mass and thus, using the stellar birth-line in \citet{1997ApJ...475..770H}~(adopting the models with protostellar surface cooling which provides lower-estimates), we can estimate the protostellar radius. From these calculations we can estimate the mass accretion rate of the protostars. The results are tabulated in Table~\ref{table:massacc}\space but are also summarized here. For the single protostar IRS3A this is straight-forward, but for the binary source IRS3B-ab, care must be taken. We adopt the two scenarios for the system configuration: 1.) the protostellar masses are equally divided (two 0.575~\solm\space protostars) and 2.) one protostar dominates the potential (one 1.15~\solm\space protostar). From Figure~3 in \citet{1997ApJ...475..770H}\space we estimate the stellar radius to be 2.5~\rsun, 2.5~\rsun, and 2~\rsun\space for stellar masses 0.575~\solm, 1.15~\solm, and 1.51~\solm, respectively. From Figure~3 in \citet{1997ApJ...475..770H}\space we estimate the stellar luminosity to be 1.9~\lsun, 3.6~\lsun, and 2.5~\lsun\space for stellar masses 0.575~\solm, 1.15~\solm, and 1.4~\solm, respectively (see Section~\ref{sec:diskmass}).

Considering the bolometric luminosities for IRS3B and IRS3A given in Section~\ref{sec:diskmass}, we find the $\dot{M}\sim4.95\times10^{-7}$~\solm~yr$^{-1}$ for IRS3A. Then for IRS3B-ab, in the first scenario (two 0.575~\solm~protostars), we find $\dot{M}\sim1.5\times10^{-6}$~\solm~yr$^{-1}$ and in the second scenario (one 1.15~\solm~protostar), we find $\dot{M}\sim6.6\times10^{-7}$~\solm~yr$^{-1}$. These accretion rates are unable to build up the observed protostellar masses within the typical lifetime of the Class 0 stage (\ab160~kyr) and thus require periods of higher accretion events to explain the observed protostellar masses. This possibly indicates the IRS3B-ab system is more consistent as an equal mass binary system. However, further, more sensitive and higher resolution observations to fully resolve out the dynamics of the inner disk are needed to fully characterize the sources.

We further compare the accretion rates derived here with a similar survey towards Class 0+I protostars \citep{2017ApJ...834..178Y}. We find IRS3A is consistent with L1489 IRS, a Class I protostar with a M$_{*}\sim1.6$~\solm\space\citep[][]{2013ApJ...770..123G}\space and a $\dot{M}\sim2.3\times10^{-7}$~\solm~yr$^{-1}$~\citep[][]{2014ApJ...793....1Y}. Furthermore, in the case IRS3B-ab is an equal mass binary, the derived accretion rates as compared with the sources in \citet{2017ApJ...834..178Y}\space are in the upper echelon of rates. However, in the case IRS3B-ab is best described as a single mass protostar, the derived accretion rates are consistent with TMC-1 and TMC-1A, other Class 0+I sources in \citet{2017ApJ...834..178Y}.


%Thus, the mass accretion estimates from the bolometric luminosity are able to efficiently transport momentum from the disk onto the stellar surface without inducing protostellar winds.

% disk transport
% IRS3B 40.1    IRS3A 50.9
% def tmp(temp): cs = (1.5 * nconstants.kb * temp * 2.0/(nconstants.mh * 2.36)) ** 0.5; return cs ** 3 / nconstants.g / nconstants.msun * nconstants.year # output in Msun/yr
% tmp(40.1)
% tmp(50.9)

% mass accretion
% Hartmann et al 1997 (1997ApJ...475..770H)
% D'Antona and Mazzitelli 1994 (1994ApJS...90..467D)
% Class 0 stage 10**6 years, compared to Yen 2017 (2017ApJ...834..178Y)
% Lstar = 1 Lsun (Mstar / 0.5msun)^0.9(Rstar/2Rsun)^2.34
% rstar * c.rsun / (c.g * mstar * c.msun) * (lacc * c.lsun) / c.msun * c.yr

% def tmp(mass, radius, lum):
%    lstar = (mass / 0.5 ) ** 0.9 * (radius/ 2.) ** 2.34
%    mdot = radius * c.rsun / (c.g * mass * c.msun) * ((lum - lstar) * c.lsun) / c.msun * c.yr
%    print(f'Radius: {radius:0.2e}')
%    print(f'Lbol: {lum:0.2e}: Lstar: {lstar:0.2e}')
%    print(f'Mdot: {mdot:0.2e}')
%    print(f'Macc in 160kyr: {mdot * 160e3:0.2e}')
%    lum /= 2
%    mdot = radius * c.rsun / (c.g * mass * c.msun) * ((lum - lstar) * c.lsun) / c.msun * c.yr
%    print(f'Lbol: {lum:0.2e}: Lstar: {lstar:0.2e}')
%    print(f'Mdot: {mdot:0.2e}')
%    print(f'Macc in 160kyr: {mdot * 160e3:0.2e}')
%    lum *= 4
%    mdot = radius * c.rsun / (c.g * mass * c.msun) * ((lum - lstar) * c.lsun) / c.msun * c.yr
%    print(f'Lbol: {lum:0.2e}: Lstar: {lstar:0.2e}')
%    print(f'Mdot: {mdot:0.2e}')
%    print(f'Macc in 160kyr: {mdot * 160e3:0.2e}')
% 
% 
% print(f'IRS3A')
% print(f'{tmp(1.513, 2., 9.2 * (288. / 230.) ** 2)}')
% print(f'IRS3B-single')
% print(f'{tmp(1.15, 2.5, 8.3 * (288. / 230.) ** 2)}')
% print(f'IRS3A-binary')
% print(f'{tmp(1.15 / 2, 2.5, 8.3 * (288. / 230.) ** 2)}')


While the currently estimated accretion rates for IRS3B and IRS3A would not be able to assemble the observed protostar masses in the lifetime of a Class 0 protostar, accretion rates are not necessarily constant through the protostellar phase. The well-known FU Orionis phenomenon are exemplary examples of non-steady accretion in protostars \citep[e.g. ][]{1996ARAA..34..207H,2014prpl.conf..387A}. Accretion bursts have also been observed in both Class I and Class 0 protostars in recent years \citep{2015ApJ...800L...5S, 2013prpl.conf1H023F}. Gravitational instability in disks has been proposed as a mechanism to drive outburts with the accretion of clumps of material from the disks \citep{2011ASPC..451..213S, 2017MNRAS.465....2M, 2014ARep...58..522V,2014MNRAS.444..887D,2020arXiv201005939S}. In this scenario, the accretion luminosity increases quickly, stabilizing the disks. After the accretion event has finished, the protostars undergo a ``quiescent'' stage while the disk can re-develop gravitational instabilities and fragment. Two possible signatures for this mechanism would be in the outflow configuration: bi-polar jets with periodically spaced knots and gravitational instabilities of the disk. IRS3B does exhibit a gravitationally unstable disk (Section~\ref{sec:stability}), but the outflow, while having many bright features,% Appendix~\ref{sec:outflow},
does not show periodically spaced knows like the example from \citet{2015Natur.527...70P}. Thus, it is possible that both IRS3B and IRS3A have undergone past accretion burst, helping to explain their current masses and relatively low inferred accretion rates, but they do not currently exhibit features of outbursting protostars and we cannot unequivocally state that they have undergone past outbursts.

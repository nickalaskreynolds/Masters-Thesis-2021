{
%\addcontentsline{toc}{section}{\protect\numberline{}Abstract}
\section*{Abstract}
Star and planet formation are the outcomes of gravitational collapse of molecular clouds and subsequent angular momentum transport mediated through the formation of protostellar/protoplanetary disks. We present new \acrfull{alma} observations towards a compact (230~\gls{au}) protostellar \gls{multiple}\space system, L1448 IRS3B. We resolve the \Gls{keplerian} rotation for both the circum-triple disk in IRS3B and the disk around IRS3A. Furthermore, we use the \gls{molecular} line kinematic data and radiative transfer modeling of the molecular line emission to confirm that the disks are in Keplerian rotation with fitted central masses of $1.19^{+0.13}_{-0.07}$ for IRS3B-ab, $1.51^{+0.06}_{-0.07}$~\msun\space for IRS3A, and place an upper limit on the central \gls{protostar} mass for the tertiary IRS3B-c of 0.2~\msun. We measure the mass of the fragmenting disk of IRS3B to be 0.29~\acrshort{msun} from the dust \gls{continuum} emission of the circum-multiple disk and estimate the mass of the clump surrounding IRS3B-c to be 0.07~\msun. We also find that the disk around IRS3A has a mass of 0.04~\msun. By analyzing the Toomre~Q parameter, we find the IRS3A circumstellar disk is gravitationally stable (Q$>$5), while the IRS3B disk is consistent with a gravitationally unstable disk (Q$<$1) between the radii 200-500~au. This coincides with the location of the spiral arms and the tertiary companion IRS3B-c, supporting the hypothesis that IRS3B-c was formed in situ via fragmentation of a gravitationally unstable disk.
}
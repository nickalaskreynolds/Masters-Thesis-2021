\section{Introduction}\label{sec:intro}
Star formation takes place in dense cores within molecular clouds \citep{1987ARAA..25...23S}, that are generally found within filamentary structures \citep{2014prpl.conf...27A}. These molecular clouds undergo gravitational collapse and due to the conservation of angular momentum will form a disk of accreting material. These protostellar disks regulate the formation of the stars and planetary systems that are embedded within. 

In particular, the Perseus Molecular Cloud, hosts a plethora of \acrlong{yso}s \citep[YSOs;][]{2014ApJ...787L..18S,2009ApJ...692..973E} and is nearby \citep[d\ab288$\pm$22~pc; e.g.,][]{2018arXiv180803499O, 2019ApJ...879..125Z}, making its protostellar population ideal for high-spatial resolution studies. By observing these YSOs during the early stages of star formation, we can learn about how cores collapse and evolve into protostellar and/or proto-multiple systems, and how their disks may form into proto-planetary systems.

Protostellar systems have been classified into several groups following an evolutionary sequence: Class 0, the youngest and most embedded objects characterized by low L$_{bol}$/L$_{submm}$ \citep[$<5\times10^{-3}$; ][]{1993ApJ...406..122A} and \tbol\space$\le$70~K, Class I sources which are still enshrouded by an envelope that is less dense than the Class 0 envelope, with T$_{bol}<=650$~K, Flat Spectrum sources, which are a  transition phase between Class I and Class II, and Class II objects, which have shed their envelope and consist of a pre-main sequence star (pre-MS) and a protoplanetary disk. Most stellar mass build-up is expected to occur during the Class 0 and Class I phases \citep[$<5\times10^{5}$~yr; e.g.][]{2018arXiv180711262K,1987IAUS..115....1L}, because by the time the system has evolved to the Class II stage, most of the mass of the envelope has been either \gls{accrete}d onto the disk/protostar or blown away by \gls{outflow}s \citep[][]{2006ApJ...646.1070A, 2014ApJ...784...61O}.

Studies of multiplicity in \gls{field star}s have observed multiplicity fractions of 63\% for nearby stars \citep[][]{1962AJ.....67R.590W}, 44-72\% for Sun-like stars \citep[][]{1983ARAA..21..343A, 2010ApJS..190....1R}, 50\% for F-G type\footnote{see \textit{Spectral types}, Section~\ref{sec:glossary}} nearby stars \citep[][]{1991AA...248..485D}, 84\% for A-type stars \citep[][]{2017ApJS..230...15M}, and 60\% for pre-MS stars \citep[][]{1994ARAA..32..465M}. These studies demonstrate the high frequency of stellar multiples and motivates the need for further multiplicity surveys toward young stars to understand their formation mechanisms.

Current theories suggest four favored pathways for forming multiple systems: \gls{turbulent} \gls{fragmentation} \citep[on scales \ab1000s of au; e.g.][]{2004ApJ...617..559P, 2004ApJ...600..769F}, \gls{thermal fragmentation} \citep[on scales \ab1000s of au; e.g.][]{2010ApJ...725.1485O, 2013ApJ...764..136B}, \gls{gi}\space within disks \citep[on scales \ab100s of au; e.g.][]{1989ApJ...347..959A, 2009MNRAS.392..413S, 2010ApJ...708.1585K}, and/or loose \gls{dynamical} of cores \citep[\ab10$^{4-5}$~au scales][]{2002MNRAS.336..705B, 2019ApJ...887..232L}. Additionally, stellar multiples may evolve via multi-body dynamical interactions which can alter their hierarchies early in the star formation process \citep{2002MNRAS.336..705B, 2010MNRAS.404..721M, 2012Natur.492..221R}. In order to fully understand star formation and multiple-star formation, it is important to target the youngest systems to characterize the initial conditions.

The \acrshort{vla} Nascent Disk and Multiplicity (\acrshort{vandam}) survey \citep{2016ApJ...818...73T} targeted all known protostars down to 20~au scales within the Perseus Molecular Cloud using the \acrfull{vla}\space to better characterize protostellar multiplicity. They found the multiplicity fraction (MF) of Class 0 protostars to be \ab57\% (15-10,000~au scales) and \ab28\% for close companions (15-1,000~au scales), while, for Class I protostars, the MF for companions (15-10,000~au scales) is 23\% and 27\% for close companions (15-1,000~au scales). This empirical distinction in MF motivates the need to observe Class 0 protostars to resolve the dynamics before the systems evolve. It was during this survey that the multiplicity of L1448 IRS3B, a compact (\ab230~au) triple system, was discovered. \citet{2016Natur.538..483T} observed this source at 1.3~mm\footnote{see \textit{continuum}, Section~\ref{sec:glossary}}, resolving spiral arms, kinematic rotation signatures in \ceo, \tco, and \htco, with strong outflows originating from the IRS3B system. 

L1448 IRS3B has a hierarchical configuration, which features an inner binary (separation 0\farcs25$\approx$75~au, denoted -a and -b, respectively) and an embedded tertiary (separation 0\farcs8$\approx$230~au, denoted -c). The IRS3B-c source is deeply embedded within a clump positioned within the IRS3B disk, thus we reference the still forming protostar as IRS3B-c and the observed compact emission as a ``\gls{clump}'' around IRS3B-c. \citet{2016ApJ...818...73T}\space found evidence for Keplerian rotation around the disks of IRS3B and IRS3A. They also found that the circum-triple disk was likely gravitationally unstable.

Theory suggests that during stellar mass assembly via disk accretion, fragmentation via gravitational instability (hereafter GI) may occur if the disk is sufficiently massive, cold, and rapidly accreting\citep{1989ApJ...347..959A, 1999ApJ...525..330Y, 2010ApJ...710.1375K}. Due to the scales of fragmentation, and on-going \gls{infall}, fragments likely turn into stellar or \gls{brown dwarf} mass companions, and GI is a favored pathway for the formation of compact multi-systems ($\lesssim$100~au). Since observations show that the youngest systems, like L1448 IRS3B, have higher disk masses than their more evolved counterparts \citep{2020ApJ...890..130T}, we would also expect observational signatures of disk instability and fragmentation to be most prevalent in the Class 0 stage.

The wide and compact proto-multiple configurations of IRS3A and IRS3B contained within a single system provides a test bed for multiple star formation pathways to determine which theories best describe this system. Here we detail our follow-up observations to \citet{2016Natur.538..483T} of L1448 IRS3B with the \acrfull{alma} in \gls{band} 7, with $2\times$~higher \gls{resolution} and $6\times$~higher \gls{sensitivity}. We resolve the kinematics toward both IRS3B and IRS3A with much higher fidelity that the previous observations, enabling us to characterize the nature of the rotation in the disks, measure the protostar masses, and characterize the stability of both disks. We show our observations of this system and describe the data reduction techniques in Section~\ref{sec:obs}, we discuss our empirical results and our use of molecular lines in Section~\ref{sec:results}, we further analyze the molecular line kinematics in Section~\ref{sec:keprotation}, we further detail our models and the results in Section~\ref{sec:kmodelresults}, and we interpret our findings in Section~\ref{sec:discsec}, where we discuss the implications of our empirical and model results and future endeavors.

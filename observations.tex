%#%#%#%#%#%#%#%#%#%#%#%#%#%#%#%#%#%#%#%#%#%#%#%#%#%#%#
\section{Observations}\label{sec:obs}

We observed L1448 IRS3B with ALMA in Band 7 (879~\micron) during Cycle 4 in two configurations, an extended (C40-6) and a compact (C40-3) configuration in order to fully recover the total \gls{flux} out to \ab5\arcsec\space \glspl{angular} in addition to resolving the structure in the disk. %C40-6, was used on 2016 October 1~and~4 with 45 antennas. The baselines ranged from 15 to 3200~meters, for a total of 4495 seconds on source (8052 seconds total) for both executions. C40-3, was used on 19 December 2016 with 41 antennas. The baselines covered 15 to 490 meters for a total of 1335 seconds on source (3098 seconds total).

The complex gain calibrator was J0336$+$3218, the bandpass calibrator was J0237$+$2848, and the flux calibrator was the monitored quasar J0238$+$1636. The observations were centered on IRS3B. IRS3A, the wide companion, is detected further out in the \gls{pb} with a beam efficiency \ab60\%).We summarize the observations in Tables~\ref{table:obssummary1}~and~\ref{table:obssummary2}. % and further detail our observations and reductions in Appendix~\ref{sec:appobs}.

%It should also be noted there is possible line blending of \lhtcn\space and \lsot\space\citep{1997Icar..130..355L} (Table~\ref{table:obssummary2}). The \sot\space line has an Einstein A coefficient of 2.4$\times10^{-4}$~s$^{-1}$ with an upper level energy of 93~K, demonstrating the transition line strength could be strong. \sot\space provides another shock tracer which could be present toward the protostars. We label \htcn\space and \sot\space together for the rest of this analysis to emphasize the possible line blending of these molecular lines. Additionally, the \co\space and \sio\space emission primarily trace outflowing material and analysis of these data is beyond the scope of this paper, but the integrated intensity maps of select velocity ranges are shown in Appendix~\ref{sec:coemission}~and~\ref{sec:sioemission}. The results of this analysis are summarized for each of the sources in Table~\ref{table:obssummary3}.

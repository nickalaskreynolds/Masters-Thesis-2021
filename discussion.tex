
%#%#%#%#%#%#%#%#%#%#%#%#%#%#%#%#%#%#%#%#%#%#%#%#%#%#%#
%#%#%#%#%#%#%#%#%#%#%#%#%#%#%#%#%#%#%#%#%#%#%#%#%#%#%#
\section{Discussion}\label{sec:discsec}

\subsection{Origin of Triple System and Wide Companion}\label{sec:origin}
%Proto-multiple systems like that of IRS3B and IRS3A can form via several possible pathways pathways: thermal fragmentation (on scales \ab1000s of au), turbulent fragmentation (on scales \ab1000s of au), gravitational instabilities within disks (on scales \ab100s of au), and/or loose dynamical capture of cores (on scales \ab10$^{4-5}$~au).
To constrain the main pathways for forming multiple systems, we must first constrain the protostellar geometrical parameters and then the (in)stability of the circum-multiple disk. Previous studies towards L1448 IRS3B \citep[see ][]{2016Natur.538..483T} achieved \ab0\farcs4\space molecular line resolution, roughly constraining the protostellar mass. The high resolution and high sensitivity data we present allows constraints on the stability of the circumstellar disk of IRS3B and sheds light on the formation pathways of the compact triple system and the wide companion. The circumstellar disk around the wide companion, IRS3A, has an orthogonal major axis orientation to the circumstellar disk of IRS3B, favoring formation mechanisms that result in wider companions forming with independent angular momentum vectors. The circumstellar disk around IRS3B is massive, has an embedded companion (IRS3B-c), and has spiral arms, which are indicative of gravitational instability, and we will more quantitatively examine the (in)stability of the disk in Section~\ref{sec:struct}.

%#%#%#%#%#%#%#%#%#%#%#%#%#%#%#%#%#%#%#%#%#%#%#%#%#%#%#
\subsection{Signatures of an Embedded Companion in Disk Kinematics}\label{sec:embeddedkinematics}

    Hydrodynamic simulations show that massive embedded companions within viscous disks should impact the Keplerian velocity pattern in a detectable manner \citep{2015ApJ...811L...5P}. \citet{2018MNRAS.480L..12P} showed the signatures of a massive companion embedded within a viscous, non-self gravitating disk.Their model observations are higher (\ab2$\times$) spectral and angular resolution, and more sensitive (\ab5$\times$) than the presented observations. They show a 10~M$_{J}$~mass source should be easily detectable with about 1000 orbits of evolution by analyzing the moment 1 maps. More recently, several studies of protoplanetary disks have confirmed these predictions of localized Keplerian velocity deviations for moderately massive planets \citep[][]{2018ApJ...860L..13P,2019NatAs...3.1109P}. However, these systems are much more evolved ($>3$~Myr), with quiescent, non-self gravitating disks, and likely experienced thousands of stable orbits compared to IRS3B, a self-gravitating and actively accreting Class 0 source, with a companion that likely has completed only a few dynamically changing orbits.
    
    \citet{2020arXiv200715686H}\space performed simulations of a viscous, self-gravitating disk (0.3~\msun) around a 0.6~\msun\space source much more similar in physical parameters or IRS3B than the types of systems discussed in the preceding paragraph. Their results showed that the effects of self gravity will provide ``kinks'' at high resolution and sensitivity. Additionally, \citet{2011IAUS..276..463V}\space showed due to exchange of momentum with the disk, or dispersal due to tidal torques, the fragment radius would be drastically changing up to an order-of-magnitude over the evolution of the disk. All of these work to mask definitive observable kinematic deviations of embedded companions in the disk.

%#%#%#%#%#%#%#%#%#%#%#%#%#%#%#%#%#%#%#%#%#%#%#%#%#%#%#
\subsection{Disk Structure}\label{sec:struct}
With the high resolutions observations, we can construct a radial profile of the continuum emission to analyze disk structure. The circumstellar disk of IRS3B has prominent spiral arms but the radial profile will azimuthally average this emission. In order to construct the radial profile, we have to define: an image center to begin the extraction, the geometry (position angle and inclination) of the source, and the size of each annuli. The system geometry and image center were all adapted from the PV diagram fit parameters and the radius of the annuli is defined as half the average synthesized beamsize \citep[Nyquist Sampling;][]{1928TAIEE..47..617N}. We then convert from flux density to mass via Equation~\ref{eq:dustmasseq}\space and further construct a disk mass surface density profile. To convert from flux density into dust mass, we adopt a radial temperature power law with a slope of -0.5, assuming the disk at 100~au can be described with a temperature of (30~K)$\times(L_{*}$/\lsun)$^{0.25}$. The temperature profile has a minimum value of 20~K, based on models of disks embedded within envelopes \citep[][]{2003ApJ...591.1049W}. While we adopt a temperature law profile, protostellar multiples are expected to complicate simple radial temperature profiles.

Towards IRS3B, in order to mitigate the effects of the tertiary source in the surface density calculations, we use the tertiary subtracted images.%, described in the Appendix~\ref{sec:tertsub}.
The system geometric parameters used for the annuli correspond to an inclination of 45\deg\space and a position angle of 28\deg. The PV/Gaussian fits were used here for ease of reproducibility and utilizing the \pdspy\space results would still be consistent. The largest annulus extends out to 5\arcsec, corresponding to the largest angular scale on which we can recover most emission. The temperature at 100~au for IRS3B-ab is taken to be $\approx$40.1~K.% We show both the extracted flux radial profile and radial surface density profile for IRS3B-ab in Figure~\ref{fig:surfacedensity}. The radial surface density profile shows a flat surface density profile out to \ab400~au.

%Towards IRS3A, the system geometry parameters used for the annuli correspond to an inclination of 69\deg\space and a position angle of 133\deg. With this method, we construct a radial surface density profile to analyze the stability of the disk (Figure~\ref{fig:irs3asurfacedensity}). The temperature at 100~au for IRS3A is taken to be $\approx$50.9~K. The circumstellar disk of IRS3A is much more compact than the circumstellar disk of IRS3B, with the IRS3A disk radius \ab150~au, and thus the assumed temperature at 100~au is a good approximation for the median disk temperature.


%#%#%#%#%#%#%#%#%#%#%#%#%#%#%#%#%#%#%#%#%#%#%#%#%#%#%#
\subsubsection{Disk Stability}\label{sec:stability}
The radial surface density profiles allow us to characterize the stability of the disk to its self-gravity as a function of radius. The Toomre~Q parameter (herein Q) can be used as a metric for analyzing the stability of a disk. It is defined as the ratio of the rotational shear and thermal pressure of the disk versus the self-gravity of the disk, susceptible to fragmentation. When the Q parameter is $<$1, it indicates a gravitationally unstable region of the disk. 

Q is defined as: 
\begin{equation}
Q = \frac{c_{s} \kappa}{\pi G \Sigma}
\end{equation}
where the sound speed is c$_s$, the epicyclic frequency is $\kappa$ corresponding to the orbital frequency ($\kappa = \Omega$ in the case of a Keplerian disk), and the surface density is $\Sigma$, and G is the gravitational constant.

We further assume the disk is thermalized and the disk sound speed radial profile is given by the kinetic theory of gases: 
\begin{equation}\label{eq:cs}
c_{s}\left(T\right) = \left(\frac{k_{b}T}{m_{H} \mu}\right) ^{0.5}
\end{equation}
where T is the gas temperature and $\mu$ is the mean molecular weight (2.37). We then evaluate the angular frequency as a function of radius,
\begin{equation}
\Omega\left(R\right) = \left(\frac{GM_{*}}{R^{3}}\right)^{0.5}
\end{equation}
 where M$_{*}=1.15$~\solm.
 

Simulations have shown that values of Q$<$1.7 (calculated in 1D) can be sufficient for self-gravity to drive spiral arm formation within massive disks while Q $\approx$1\space is required for fragmentation to occur in the disks \citep[][]{2010ApJ...710.1375K}. Figure~\ref{fig:irs3btoomreq} shows the Q radial profile for the circumstellar disk of L1448 IRS3B, which varies by an order of magnitude across the plotted range (0.4-4). The disk has Q$<1$\space and therefore is  gravitationally unstable starting near \ab120~au, interior to the location of the embedded tertiary within the disk and extending out to the outer parts of the disk (\ab500~au) as indicated by the IRS3B Q radial profile. The prominent spiral features present in the circumstellar disk span a large range of radii (10s-500~au).

%Figure~\ref{fig:irs3btoomreq} shows the Q radial profile for the circumstellar disk of L1448 IRS3A. The IRS3A dust continuum emission, while having possible spiral arm detection (Figure~\ref{fig:contimage}), is more indicative of a gravitationally stable disk through the analysis of the Q radial profile (Q$>$5 for the entire disk). This is due to the higher mass central protostar and lower disk surface density, as compared to the circumstellar disk of IRS3B. Thus substructures in IRS3A may not be gravitationally driven spiral arms and could reflect other substructure. The circumstellar disk around IRS3A has a mass of 0.04~\solm\space and the protostar has a mass of 1.4~\solm.

%%%%%%%%%%%%%%%%%%%%%%%%%%%%%%%%%%%%%%%%%%%%%%%%%%%%%%%%%%

\subsection{Interpretation of the Formation Pathway}
%\subsubsubsection{Wide Companion: IRS3A and IRS3B systems}
The formation mechanism for the IRS3A source, IRS3B system as a whole, and the more widely separated L1448 NW source \cite{2016ApJ...818...73T}, is most likely turbulent fragmentation, which works on the 100s-1000s~au scales \citep[][]{2010ApJ...725.1485O, 2019ApJ...887..232L}. Companions formed via turbulent fragmentation are not expected to have similar orbital configurations and thus are expected to have different V$_{sys}$, position angle, inclination, and outflow orientations. For the wide companion, IRS3A, the disk and outflows are nearly orthogonal to IRS3B and have different system velocities (e.g., 5.3~km~s$^{-1}$\space and 4.9~km~s$^{-1}$, respectively; Tables~\ref{table:pvtable}~and~\ref{table:pdspykinematic}). \citet{2019ApJ...884....6M} has shown protostellar systems dynamically ejected from multi-body interactions are less likely to be disk bearing. Considering the low systemic velocity offset (IRS3A: 5.3~km~s$^{-1}$, IRS3B: 4.9~km~s$^{-1}$), the well ordered Keplerian disk of IRS3A, and relative alignment along the long axis of the natal core \citep{2017MNRAS.469.3881S}, the systems would not likely have formed via the dynamical ejection scenario from the IRS3B system \citep[][]{2012Natur.492..221R}.

%\subsubsubsection{Triple System: IRS3B-abc}
In contrast, the triple system IRS3B appears to have originated via disk fragmentation. The well organized \cso\space emission, which traces the disk continuum emission, indicates that the circum-multiple disk in IRS3B is in Keplerian rotation at both compact and extended spatial scales (0\farcs2 to $>$2\farcs0; 50~au to $>$600~au) (see Figure~\ref{fig:irs3bc17omoment}). The derived disk mass (M$_{d}$/M$_{s}\sim25$\%) is high, such that the effects of self-gravity are important \citep[][]{1990ApJ...358..515L}. The low-m (azimuthal wavenumber) spiral arms observed in the disk are consistent with the high mass \citep{2016ARAA..54..271K}. The protostellar disk is provided stability on scales near the central potential due to the shear effects of Keplerian rotation and higher temperatures, while, at larger radii, the rotation velocity falls off and the local temperature is lower, allowing for local gravitational instability.  Moreover, as seen in Figure~\ref{fig:irs3btoomreq}, Q falls below unity at radii $>120$~au, coincident with the spatial location  of the tertiary IRS3B-c, as expected if recently formed via gravitational instability in the disk. Additionally the inner binary, IRS3B-ab, could have formed via disk fragmentation prior to the IRS3B-c, resulting in the well-ordered kinematics surrounding IRS3B-ab.

The PV analysis of IRS3B-c also suggests that the central mass of the tertiary continuum source is low enough ($\sim0.02$~\solm) to not significantly alter the kinematics of the disk (Figures~\ref{fig:irs3bc17omoment}~and~\ref{fig:irs3abc17omoment1})

The apparent co-planarity of IRS3B-abc and the well-organized kinematics of both the disk and envelope tracers, \cso\space and \htcop, argue against the turbulent fragmentation pathway within the subsystem. The PV-diagram of IRS3B-ab is well structured in various disk tracing molecules and the outflow orientation of IRS3B-c is aligned with the angular momentum vector of IRS3B-ab, making dynamical capture unlikely.

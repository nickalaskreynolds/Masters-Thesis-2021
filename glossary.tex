{

\newglossaryentry{field star}
{
        name=field star,
        description={A star randomly situated along the line-of-sight to the field being observed. These are not associated with the desired system and thus need to be characterized to avoid contamination.}
}
\newglossaryentry{core}
{
        name=core,
        description={A concentrated spherical region of dust and gas that lacks  primary internal heating source such as a protostar or star-like object. These are typically cold and emit in the far-IR and sub-mm wavelength radio regime.}
}
\newglossaryentry{flux}
{
        name=flux,
        description={Measurement of energy per area, expressed in ergs~m$^{-2}$. Flux density, a measurement of energy per area for a given frequency, typically measured in Jy~Hz$^{-1}$. Intensity, a measurement of the flux density per unit solid angle, typically measured in Jy~beam$^{-1}$~Hz$^{-1}$. }
}
\newglossaryentry{infall}
{
        name=infall,
        description={The process by which material dynamically moves in the disk. Typically on scale of the envelope, this processes happens isometrically, however, due to the conservation of angular momentum, closer to the protostar this process happens along magnetic field lines at the poles.}
}
\newglossaryentry{clump}
{
        name=clump,
        description={A region of concentration emission that may undergo contraction and form a protostar.}
}
\newglossaryentry{accrete}
{
        name=accrete,
        description={The mass transfer from one body to another, typically in the form of dust and gas into a large central body. This process does not happen with 100\% efficiency and (assuming the system is virialized) about half of the energy will go into the central source while the rest will heat up and radiate the surrounding material.} % https://www.astro.umd.edu/~richard/ASTR680/NS_lec_2.pdf
}
\newglossaryentry{outflow}
{
        name=outflow,
        description={Gas and dust that is typically ejected from the poles (perpendicular to the rotation axis) central gravitating source at ``high'' (10s of \kms) velocities. This is theorized to result from accretion events. These outflows control the stellar mass accretion rate as the surrounding mass reservoir is evacuated due to the highly energetic steam.}
}
\newglossaryentry{bol}
{
        name=bolometric,
        description={The bolometric luminosity is the total luminosity of an object, integrated over all wavelengths. The bolometric temperature is the corresponding temperature of a radiating blackbody (effective temperature) to produce the equivalent bolometric luminosity.}
}
\newglossaryentry{blackbody}
{
        name=blackbody,
        description={A body emitting uniformly according to Planck's Law (solely dependent on temperature).}
}
\newglossaryentry{type}
{
        name=spectral type,
        description={Characterization of a star based on the strength of the atomic/moelcular line transitions in the spectrum which objectively describe the photospheric surface. These are classified into: O, B, A, F, G, K, M, L, and T groups (from bright blue giants to dim red dwarfs).}
}
\newglossaryentry{spectra}
{
        name=spectrum,
        description={A plot of intensity versus wavelength (or equivalent unit) which can characterize the composition of the emitting body.}
}
\newglossaryentry{brown dwarf}
{
        name=brown dwarf,
        description={Sub-stellar object (not massive enough to begin fusion reactions) between 18-80~M$_{J}$. These are the most massive planet-like objects and least massive stellar-like objects.}
}
\newglossaryentry{continuum}
{
        name=continuum,
        description={Radiation that is emitted into a broad range of wavelengths and is not associated with emission of a single atomic/molecular line transition. The resulting spectra appears smooth. This is primarily generated through thermal radiation.}
}
\newglossaryentry{protostar}
{
        name=protostar,
        description={A young pre-main sequence type object till enveloped in an envelope of gas and dust. The precursor to a hydrogen-fusing, main-sequence star.}
}
\newglossaryentry{multiple}
{
        name=multiple,
        description={A system of objects that constitutes gravitational companions at both short and wide distance scales.}
}
\newglossaryentry{au}
{
        name=astronomical unit,
        description={The average distance between the Earth and the Sun, roughly 150 million kilometers}
}
\newglossaryentry{molecular}
{
        name=molecular line,
        description={Radio wavelength emission that comes from transitions between different  (rotational and/or vibrational) states of the molecule and have wavelengths normally in the submillimeter and millimeter wavelength range.}
}
\newglossaryentry{keplerian}
{
        name=Keplerian rotation,
        description={Motion of a celestial body with respect to another which forms a 2-D orbital plane.}
}
\newglossaryentry{yso}
{
        name=YSO,
        description={Young stellar object; a star in the earliest stages of evolution, typically referred to as protostars (the youngest, more embedded objects) or pre-main sequence stars.}
}
\newglossaryentry{coordinates}
{
        name=spherical celestial coordinates,
        description={A spherical coordinate system based on the celestial bodies such that 0\deg\space declination marks the equitorial plane and 0\deg\space right ascension marks the great circle that passes through the Earth's North and South pole at vernal equinox.}
}


\newglossaryentry{turbulent}
{
        name=turbulent fragmentation,
        description={Density fluctuations brought about by chaotic velocity field changes, capable of supporting the cloud against gravitational collapse while simultaneously producing regions that exceed Jean's Criteria.}
}
\newglossaryentry{fragmentation}
{
        name=fragmentation,
        description={Runaway accretion process by which a region of gas and dust will collapse to form self-gravitating cores of higher densities. This can occur via thermal Jean's fragmentation, turbulent fragmentation, and gravitational instability.}
}

\newglossaryentry{thermal fragmentation}
{
        name=thermal fragmentation,
        description={Source of support against gravity or density enhancement due to the thermal properties of the gas, generally driven by a heating source.}
}

\newglossaryentry{gi}
{
        name=gravitational instability,
        description={A process by which  a fragmenting disk forms cores. Generally thought to occur when the local density of the gas exceeds the energy driven by the central gravitating source.}
}

\newglossaryentry{dynamical}
{
        name=dynamical capture,
        description={The process by which wide companions may form, where external gravitating bodies become gravitationally bound to another body. This process is unlikely and thus is more prominent in high-density regions.}
}


\newglossaryentry{angular}
{
        name=angular scale,
        description={The distance in the spherical celestial coordinates between two regions in the sky. This is measured in units on degree (or equivalent) but can be converted into a linear distance via \textit{in-plane distance between sources = angular scale in radians x distance to sources} }
}

\newglossaryentry{pb}
{
        name=primary beam,
        description={The maximum recoverable region for a given field-of-view and can be thought of as antenna or array response to different points in the sky. For heterogenous arrays, generally thought to be uniform for each of the individual antennas, but can be more complicated.}
}


\newglossaryentry{band}
{
        name=band,
        description={A standardized, broad selection of wavelengths for either a filter type or a given facility. The ALMA facility has 8 bands (3-10) which span wavelengths 100~GHz to 950~GHz.}
}

\newglossaryentry{resolution}
{
        name=resolution,
        description={The minimal recoverable size of an observation. Objects small than this size are considered unresolved and thus only total intensity is able to be recovered. It is equivalent to $R \sym\lambda / D_{eff}$. For antenna arrays, D$_{eff}$\space is determined by the largest distance between a pair of antenna and for a single telescope D$_{eff}$\space is determined by the size of the aperture.}
}

\newglossaryentry{sensitivity}
{
        name=sensitivity,
        description={The minimum recoverable signal a telescope can distinguish above the background noise level. This is dominated by the system temperature, the number of antenna, the integration time, the integrated wavelength, and the correlator efficiency.}
}
\newglossaryentry{kinematic}
{
        name=kinematic,
        description={The study of the motion of celestial bodies irrespective of the underlying physical phenomena that causes the motion.}
}


\newacronym{alma}{ALMA}{Atacama Large Millimeter/submillimeter Array}

\newacronym{msun}{\msun}{Solar masses}
\newacronym{vla}{VLA}{Karl G. Jansky Very Large Array}
\newacronym{vandam}{VANDAM}{VLA and ALMA Nascent Disk and Multiplicity survey}
\newacronym{ra}{RA}{Right ascension; see Glossary \gls{coordinates}}
\newacronym{dec}{Dec}{Declination; see Glossary \gls{coordinates}}
\newacronym{snr}{S/N}{Signal-to-noise ratio; ratio between the peak signal and the noise level (see Glossary \gls{sensitivity})}




}
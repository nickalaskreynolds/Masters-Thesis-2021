
\section{Keplerian Rotation}\label{sec:keprotation}

To determine the stability of the circumstellar disks around IRS3B and IRS3A, the gravitational potentials of the central sources must be constrained. The protostars are completely obscured at $\lambda\space<\space3$~\micron, rendering spectral typing impossible and kinematic measurements of the protostar masses from disk rotation are required to characterize the protostars themselves. Assuming the gravitational potential is dominated by the central protostellar source(s), one would expect the disk to follow a Keplerian rotation pattern if the rotation velocities are large enough to support the disk against the protostellar gravity. These Keplerian motions will be observed as Doppler shifts in the emission lines of molecules due to their relative motion within the disk. Well-resolved disks with Keplerian rotation are observed as the characteristic ``butterfly'' pattern around the central gravitational potential: high velocity emission at small radii to low velocity emission at larger radii, and back to high velocity emission at small radii on opposite sides of the disk \citep[e.g., ][]{2013ApJ...774...16R, 2018AA...609A..47P}.

\subsection{PV Diagrams}
To analyze the kinematics of these sources, we first examine the moment 0 (integrated intensity) maps of the red- and blue-Doppler shifted \cso\space emission to determine if the emission appears well ordered (Figure~\ref{fig:irs3bc17omoment}).% and consistent with \htcop\space(Figure~\ref{fig:h13copmomentc17o}).
We then examine the sources using a position-velocity (PV) diagram which collapses the 3-D nature of these data cubes (RA, DEC, velocity) into a 2-D spectral image. We specify the number of integrated pixels across the minor axis to limit bias from the large scale structure of the envelope and select emission originating from the disk. This allows for an estimation of several parameters via examining the respective Doppler shifted components. 

\subsubsection{IRS3B}
The PV diagrams for IRS3B are generated over a 105 pixel (2\farcs1) width strip at a position angle 28\deg. The PV diagram velocity axis is centered on the system velocity of 4.8~km~s$^{-1}$ \citep[][]{2016Natur.538..483T} and spans $\pm$5~km~s$^{-1}$ on either side, while the position axis is centered just off of the inner binary, determined to be the kinematic center, and spans 5\arcsec\space(\ab1500~au) on either side.

\cso\space appears to trace the gas within the disk of IRS3B on the scale of the continuum disk (Figure~\ref{fig:irs3bc17omoment}). It is less abundant and therefore less affected by outflow emission. We use it as a tracer for the kinematics of the disk (PV-diagram indicating Keplerian rotation; Figure~\ref{fig:l1448irs3b_c17o_pv}). The \cso\space emission extends to radii beyond the continuum disk, likely extending into the inner envelope of the protostar, while the \htcop\space emission %~(Figure~\ref{fig:h13copmomentc17o})
appears to trace larger scale emission surrounding the disk of IRS3B and emission within the spatial scales of the disk has lower intensity. This is indicative of emission from the inner envelope as shown by the larger angular scales the emission extends to with respect to \cso\space (\htcop\space PV-diagram; Figure~\ref{fig:l1448irs3b_c17o_pv}). Finally, the blended molecular line, \htcn/\sot\space appears to trace shocks in the outflows and not the disk kinematics for IRS3B. For these reasons, we do not plot the PV diagram of \htcn/\sot.

%\subsubsection{IRS3A}
%The PV diagrams for IRS3A are generated with a 31 pixel (0\farcs62) width strip at a position angle 133\deg. \cso\space is faint and diffuse towards the IRS3A disk (Figure~\ref{fig:l1448irs3a_cso_pv}) but still traces a velocity gradient consistent with rotation (Figure~\ref{fig:irs3ac17omoment}) and has a well ordered PV diagram (Figure~\ref{fig:l1448irs3a_cso_pv}). \htcn/\sot, (Figure~\ref{fig:irs3ah13cnmoment}), appears to trace the kinematics of the inner disk due to the compactness of the emission near the protostar and the appearance within the disk plane (Figure~\ref{fig:l1448irs3a_h13cn_pv}). The velocity cut is centered on the system velocity of 5.4~km~s$^{-1}$ and spans 6.2~km~s$^{-1}$ on either side. The emission from the blended \htcn/\sot\space is likely dominated by \htcn\space instead of \sot, due to the similar system velocity that is observed. \sot\space would have \ab1.05~km~s$^{-1}$\space offset which is not observed in IRS3A.

%Similar to IRS3B, the \htcop\space emission likely traces the inner envelope, indicated in Figure~\ref{fig:irs3ah13copmoment}, as it extends well beyond the continuum emission but still traces a velocity gradient consistent with rotation (Figure~\ref{fig:l1448irs3a_h13cop_pv}). The circumstellar disk emission is less resolved, however, due to the compact nature of the source and has lower sensitivity to emission because it is located \ab8~arcsec (beam efficiency\ab60\%) from the primary beam center. 

%#%#%#%#%#%#%#%#%#%#%#%#%#%#%#%#%#%#%#%#%#%#%#%#%#%#%#
\subsection{Protostar Masses: Modeling Keplerian Rotation}
The kinematic structure, as evidenced by the blue- and red-shifted integrated intensity maps (e.g., Figure~\ref{fig:irs3bc17omoment}) indicate rotation on the scale of the continuum disk. The disk red- and blue-emission emission are oriented along the disk major axis and and not along the disk minor axis, which would be expected if the emission was contaminated by outflow kinematics. We first determined the protostellar mass by analyzing the PV diagram to determine regions indicative of Keplerian rotation. We summarize the results of our PV mass fitting in Table~\ref{table:pvtable}. PV diagram fitting provides a reasonable measurement of protostellar masses in the absence of a more rigorous modeling approach. The Keplerian rotation-velocity formula, $V(R) = (GM/R)^{0.5}$ allows several system parameters to be constrained: system velocity, kinematic center position, and protostellar mass (There is a degeneracy between mass determination and the inclination angle of the Keplerian disk. We account for inclination in fitting the mass using the constraint from the major and minor axis ratio of the continuum emission.

\subsubsection{IRS3B-\lowercase{ab}}
When calculating the gravitational potential using kinematic line tracers, one must first define the position of the center of mass.  For circum-multiple systems, the center of mass is non trivial to measure, because it is defined by the combined mass of each object and the distribution can be asymmetric. Figure~\ref{fig:kincenter} compares various ``kinematic centers'' for the circumstellar disk of IRS3B depending on the methodology used. First, by fitting the midpoint between highest velocity \cso\space emission channels, where both red and blue emission is present, for IRS3B-ab using the respective red and blue-shifted emission, the recovered center is 03$^{h}$25$^{m}$36.32$^{s}$\space 30\deg45\arcmin14\farcs92 which is very near IRS3B-a. The second method, fitting symmetry in the PV-diagram, however, requires a different center in order to reflect the best symmetry of the emission arising from the disk, at 03$^{h}$25$^{m}$36.33$^{s}$\space 30\deg45\arcmin15\farcs04 which corresponds to a position north-east of the binary pair, which is close to a region of reduced continuum emission (``deficit'' in Figure~\ref{fig:zoomincont}). The first method of fitting the highest velocity emission assumes these highest velocity channels correspond to regions that are closest to the center of mass and the emission is symmetric at a given position angle. We chose the \cso\space molecule, which is not affected by the strong outflows, appears to trace the continuum disk the best, and has no outflow contamination, for fitting. The second method of fitting the PV-diagram center assumes the source is symmetric and well described by a simple Keplerian disk across the position angle of the PV cut, ignoring the asymmetry along the minor axis. Finally, we include two other positions corresponding to the peak emission in the highest velocity blue- and red- Doppler shifted channels, respectively.  Unsurprisingly, these positions are on either side of the peak fit. The difference in the position of the kinematic centers is within \ab2 resolution elements of the \cso\space map and does not significantly affect our mass determination, as demonstrated in our following analysis.

We use a method of numerically fitting the \cso\space PV diagrams employed by \citet[][]{2018ApJ...860..119G}\space and \citet[][]{2016MNRAS.459.1892S}, by fitting the emission that is still coupled to the disk and not a part of the envelope emission. This helps to provide better constraints on the kinematic center for the Keplerian circum-multiple disk. This was achieved by extracting points in the PV-diagram that have emission 10~$\sigma$\space along the position axis for a given velocity channel and fitting these positions against the standard Keplerian rotation-velocity formula. The Keplerian velocity is the max velocity at a given radius but each position within a disk will include a superposition of lower velocity components due to projection effects.

The fitting procedure was achieved using a Markov Chain Monte Carlo (MCMC) employed by the Python MCMC program  \textit{emcee} \citep{2013PASP..125..306F}. Initial prior sampling limits of the mass were set to 0.1-2~\solm. Outside of these regimes would be highly inconsistent with prior and current observations of the system. Uncertainty in the distance (22~pc) from the \textit{Gaia} survey \citep[][]{2018arXiv180803499O} and an estimate of the inclination error (10\deg) were included while the parameters (M$_{\*}$ and V$_{sys}$) were allowed to explore phase space. These place approximate limits to the geometry of the disk. The cyan lines in Figure~\ref{fig:l1448irs3b_c17o_pv}\space trace the Keplerian rotation curve with M$_{\*}$=1.15~\solm~with 3-$\sigma$~uncertainty$=0.09$~\solm, which fits the edge of the \cso\space emission from the source. This mass estimate describes the total combined mass of the gravitating source(s). Thus if the two clumps (IRS3B-a and -b) are each forming protostars, this mass would be divided between them. However with the current observations, we cannot constrain the mass ratio of the clumps. Thus, we can consider two scenarios% (Section~\ref{sec:massacc})
, an equal mass binary and a single, dominate central potential.

The \htcop\space PV-diagram (Figure~\ref{fig:l1448irs3b_c17o_pv}) shows high asymmetry emission towards the source. However, the \htcop\space emission is still consistent with the central protostellar mass measured using \cso\space emission of 1.15~\solm\space(indicated by the white dashed line). This added asymmetry is most likely due to \htcop\space emission being dominated by envelope emission, in contrast to the \cso\space being dominated by the disk. There is considerably more spatially extended and low velocity emission that extends beyond the Keplerian curve and cannot be reasonably fitted with any Keplerian curve. Additionally, there is a significant amount of \htcop\space emission that is resolved out near line-center, appearing as negative emission, whereas, the \cso\space emission did not have as much spatial filtering as the \htcop\space emission.

%\subsubsection{IRS3B-\lowercase{c}}
%We also analyzed the \cso\space kinematics near the tertiary, IRS3B-c, to search for indications of the tertiary mass influencing the disk kinematics. In Figure~\ref{fig:l1448irs3b_c17o_pv_tert}, we show the PV diagram of \cso\space within a 2\farcs0 region centered on the tertiary and plot velocities corresponding to Keplerian rotation at the location of IRS3B-c within the disk, to provide an upper bound on the possible protostellar mass within IRS3B-c. Emission in excess of the red-dashed lines could be attributed to the tertiary altering the gas kinematics. The velocity profile at IRS3B-c shows no evidence of any excess beyond the Keplerian profile from the main disk, indicating that it has very low mass. Based on the non-detection, we can place upper limits on the mass of the IRS3B-c source of $<$0.2~\solm\space as shown by the white dotted lines in Figure~\ref{fig:l1448irs3b_c17o_pv_tert}. A protostellar mass much in excess of this would be inconsistent with the range of velocities observed.

%\subsubsection{IRS3A}
%For the IRS3A circumstellar disk, the dense gas tracers \htcn\space and \cso\space were used to analyze disk characteristics and are shown in Figures~\ref{fig:l1448irs3a_cso_pv}~and~\ref{fig:l1448irs3a_h13cn_pv}. The position cut is centered on the continuum source (coincides with kinematic center), and spans 2\arcsec (\ab576~au) on either side. This provides a large enough window to collect all of the emission from the source. The dotted white lines show the Keplerian velocity corresponding to a M$_{\*}$=1.4~\solm\space central protostar which is consistent with the PV diagram.

%The spatial compactness of IRS3A limits the utility of the \htcn\space PV-diagram with the previous MCMC fitting routine. We found evidence of rotation in this line tracer from the velocity selected moment 0 map series and PV-diagram. However, from the PV diagram alone, strong constraints cannot be determined due to the compactness of the \htcn\space emission and the low S/N of \cso.

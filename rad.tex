
%#%#%#%#%#%#%#%#%#%#%#%#%#%#%#%#%#%#%#%#%#%#%#%#%#%#%#
%and \citet{2017ApJ...851...45S} for the continuum modeling
\section{Application of Radiative Transfer Models}\label{sec:kmodelresults}
To further analyze the disk kinematics, we utilize the methods described in \citet{2019ApJ...874..136S} % and further described in Appendix~\ref{sec:apppdspy}
for modeling the molecular line emission presented thus far. The modeling framework uses RADMC-3D \citep{2012ascl.soft02015D} to calculate the synthetic channel maps using 2D axisymmetric radiative transfer models in  the limit of local local thermodynamic equilibrium (LTE)  and GALARIO \citep{2018MNRAS.476.4527T} to generate the model visibilities from those synthetic channel maps. We sample the posterior distributions of the parameters to provide fits to the visibilities by utilizing a MCMC approach 
\citep[\pdspy; ][]{2019ApJ...874..136S}. \pdspy\space uses the full velocity range given by the frequency limit of the input visibilities in modeling.

%Some of the parameters are less constrained than others due to asymmetry of the disks and discussion of these parameters fall outside the scope of the kinematic models sought in this paper. Our focus for the kinematic models are: position angle (p.a.), inclination (inc.), stellar mass (M$_{*}$), disk radius (R$_D$), and system velocity (V$_{sys}$). We provide a summary of our model results in Table~\ref{table:pdspykinematic}. 

%The combined fitting of the models is computationally expensive (fitting 200 models simultaneously per ``walker integration time-step''), requiring on average $1-2\times10^{4}$\space core-hours per source to reach convergence. We run these models across 5 nodes with 24~cores/nodes each for \ab150~hours on the OU (University of Oklahoma) Supercomputing Center for Education and Research supercomputers (OSCER) to reach sufficient convergence in the parameters. The convergence state is determined when the \textit{emcee} ``walkers'' reach a steady state solution where the ensemble of walkers is not changing by an appreciable amount, simply oscillating around some median value with a statistical variance.

\subsection{IRS3B}
The \pdspy\space kinematic flared disk model results for IRS3B are shown in Figure~\ref{fig:c17o_res}\space with the Keplerian disk fit compared to the data. The system velocity fitted is in agreement with the PV-diagram analysis. There is some uncertainty in the kinematic center, due to the diffuse, extended emission near the system velocity ($<$1 km s$^{-1}$) which yields degeneracy when fitting. The models yielded similar stellar masses as compared to the PV/Gaussian fitting (3-$\sigma$~uncertainties~listed, \pdspy\space 1.19$^{+0.13}_{-0.07}$~\solm; PV: 1.15$^{+0.09}_{-0.09}$~\solm), similar position angles (\pdspy:\space 27\deg$^{+  1.8}_{-  2.9}$; PV:\ab28\deg), and while the inclinations are not similar (\pdspy:\space66\deg$^{+ 3.0}_{- 4.6}$; Gaussian:\ab45\deg), this discrepancy in inclination is most likely due to a difference in asymmetric gas and dust emission. With the tertiary subtraction method%(Appendix~\ref{sec:tertsub})
, we gaussian fit the dust continuum of IRS3B-c to preserve the underlying disk structure, then fit the IRS3B-ab disk with a single gaussian. Using the PV-diagram fitting, we attempt to fit symmetric Keplerian curves to the PV-diagram. \pdspy\space attempts to also fit the asymmetric southeast side of the disk, which is an asymmetric feature, with the model symmetric Keplerian disk. Upon further inspection of the residual map, there is significant residual emission on the south-eastern side of the disk which is likely a second order effect in the fit, however it is confined spatially and spectrally and should not have a major effect on the overall fit.

%\subsection{IRS3A}
%The \pdspy\space kinematic flared disk model results for IRS3A are shown in Figure~\ref{fig:h13cn_res}, primarily fitting the inner disk. The models demonstrate the gas disk is well represented by a truncated disk with a maximum radius of the disk of \ab40~au (most likely due to the compact nature of the emission). This disk size of 40~au is smaller than the continuum disk and results from the compact emission of \htcn. The models find a system velocity near 5.3~km~s$^{-1}$\space in agreement with the PV-diagram. The system velocity of numerous molecules (\htcop, \cso, and \htcn) are in agreement and thus likely tracing the same structure in the system. The models yielded a similar stellar mass ($1.51^{+0.06}_{-0.07}$~\solm, 3-$\sigma$~uncertainties~listed) to the estimate from the PV-diagram. Also the disk orientation of inclination (69\deg) and position angle (\ab122\deg) agree with the estimate from the continuum Gaussian fit.
